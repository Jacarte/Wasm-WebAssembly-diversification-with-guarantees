\subsection{WebAssembly Runtime Structure}
\label{def:state}

The WebAssembly Runtime section of the WebAssembly specification \cite{webassembly_runtime} describes the runtime structure of a WebAssembly program. It covers the runtime components and their interactions during the execution of a WebAssembly program. Its 10 key elements are:

\begin{enumerate}
    \item Store(St): The WebAssembly store represents the global state and is a collection of instances of functions, tables, memories, and globals. Each of these instances is uniquely identified by an address, which is usually represented as an i32 integer.

    \item Stack(S): The stack structure holds both values and control frames. 
    Values are of types \texttt{i32,i64,f32,f64}, in this paper we annotate value symbols with lower case characters, e.g., $S=[...,v,...]$..
    Control frames are used to handle block instructions, loops, and function calls.
    In this paper, we annotate control frames in the stack with upper case characters, e.g., $S=[...,A,...]$.

    \item Locals(L): Locals are mutable variables that are local to a specific function invocation. They hold values of types: \texttt{i32, i64, f32, f64}.

    \item Module Instances(Mi): A module instance is a runtime representation of a loaded and initialized WebAssembly module. It contains the runtime representation of all the definitions within a module, including functions, tables, memories, and globals, as well as the module's exports and imports.

    \item Function Instances(Fi): A function instance is a closure, which is the pairing of a function's code with a module instance. This pairing is required because the function's code might refer to other definitions within the module instance, such as globals, tables, or memories.

    \item Table Instances(Ti): A table instance is a vector of function elements. WebAssembly tables are used to support indirect function calls. For example, they abstract high-levle language features like dynamic dispatch and function pointers.

    \item  Memory Instances(M): A memory instance is a contiguous array of bytes, representing the linear memory of a WebAssembly program. 

    \item  Global Instances(G): A global instance is a global variable with a value and a mutability flag, indicating whether the global can be modified or is immutable.

    \item Export Instances(E): Export instances represent the functions, tables, elements, globals or memories that are exported by a module. 

    \item Import Instances(I): Import instances represent the functions, tables, elements, globals or memories that are imported into a module from the host. 
\end{enumerate}

In this paper, we utilize the aforementioned ten components to define the state of a Wasm program during its execution. 
We refer to the runtime state as a vector $[St, S, L, Mi, Fi, Ti, M, G, E, I]$, representing the Store, Stack, Locals, Module Instances, Function Instances, Table Instances, Memory Instances, Global Instances, Export Instances, and Import Instances, respectively.


The following sequence of runtime structures are obtained for a potential program and its modification using an if swap transformation. A and B represent control flow labels for the start of code blocks A and B respectively:

Original program:

\begin{enumerate}
    \item $[St, [S: B, A, c], L, Mi, Fi, Ti, M, G, E, I]$ the condition $c$ is in the top of the stack.
    \item $[St, [S: B, A], L, Mi, Fi, Ti, M, G, E, I]$
    If $v$ is true the program will use the jump label A (the consequence of the if contruction).
    \item $[St, [N], L, Mi, Fi, Ti, M, G, E, I]$\\
    The alternative label is discarded and the program ended up with stack N.
\end{enumerate}

Variant program:

\begin{enumerate}
    \item $[St, [S: A, B, c == 0], L, Fi, Ti, M, G, E, I]$\\ the original condition $c$ is negated and pushed back on the stack.
    \item $[St, [S: A], L, Mi, Fi, Ti, M, G, E, I]$
    By construction, the jump labels are swapped. Since the condition is negated, the first jump label is dropped and the program continues to jump label A. 
    \item $[St, [N], L, Mi, Fi, Ti, M, G, E, I]$\\
    The alternative label is discarded and the program ended up with stack N.
\end{enumerate}



The following sequence of runtime structures are obtained for a potential program and its modification using the loop unrolling transformation. A, B, A' and B' represent control flow labels for the start of the corresponding code blocks:


Original program:
\begin{enumerate}
    \item $[St, [S: N,B,A, v.., A], L, Mi, Fi, Ti, M, G, E, I]$ Entering the loop.
    \item $[St, [S: N,B,A, v], L, Mi, Fi, Ti, M, G, E, I]$ The condition $v$ for continuing the loop is evaluated from the top of the stack, if true the program jumps to label A again and the loop iterates again. Otherwise, the block starting with label B executes.
    \item $[St, [`n], L, Mi, Fi, Ti, M, G, E, I]$\\
    The condition is false, the blocks starting with label B executes, the loop ends, the state of the stack is $N$.
\end{enumerate}

Variant program:

\begin{enumerate}
    \item $[St, [S: N B',A', v.., A', B'v..A'], L, Mi, Fi, Ti, M, G, E, I]$ Entering the loop.
    \item $[St, [S: N B',A', v.., A', B'v], L, Mi, Fi, Ti, M, G, E, I]$ The condition $v$ for continuing the loop is evaluated from the top of the stack. In this case, continuing the loop means to jump to A' in the loop construction. Otherwise, if the condition is false, the block starting with label B' is executed, unconditionally breaking, as the dashed arrows in the example shows.
    \item $[St, [N], L, Mi, Fi, Ti, M, G, E, I]$\\
    After any of B' codes executes after the loop, the stack state is $N$.
\end{enumerate}



\todo{not sure we want to talk about deoptimization, IMHO rules + egraphs is enough. can we move the content here as rewrite rule above? (or we simply remove this subsection)}
Usually the concepts of rewriting rules and e-graphs are used to optimize the code, i.e. providing better code in terms of size.
We use both sides of the coin while enabling deoptimization \cite{deoptimization}. 
\tool considers all possible replacements and not only those that could lead to optimized Wasm variants.
This phenomenon could be appreciated in the example in \autoref{alg}, where infinitely large codes can be generated.
Deoptimization enables \tool to generate thousands of variants out of a single Wasm binary in a couple of minutes.
The concept of deoptimization is implicit in the proposed traversal algorithm (Algorithm \ref{alg}).
%On the other hand, this can be also controlled through rewriting rules conditions, e.g. not allowing the generation of large code.

We implement deoptimization in \tool by extending the rewriting rules with custom operands and operators.
One example of a custom operator is a special instruction wrapper, \texttt{(container ...)}, which can have an arbitrary number of instructions as operands, i.e., Wasm code appends its operands one after the other in the bytecode. 
This custom operator allows us to have rewriting rules, such as: \texttt{(x, (container x nop))}, which set any node in the e-graph to be equivalent to the same node followed by a \texttt{nop} instruction.
Thus, this concrete rewriting rule allows us to mutate Wasm binaries by inserting arbitrary \texttt{nop} instructions \cite{6494997, 10.1145/2086696.2086702}.
Therefore, easily extending this fine-grained transformation benefits to an input Wasm binary.



% Describing the WBC POC
\pocd:  Differential Computing Analysis(DCA) is considered as a side channel attack even though it actually reads the process memory traces \cite{bos2016differential}.
Bos and colleagues developed a tool, daredevil, based on the principles of the DCA attack and with it manage to break many of the publicly available white box challenges found online.
We ported the CHES2016 challenge to \Wasm, making it possible to fully exfiltrate the key by using the daredevil tool.


We apply our tool, \tool, to the original POC binary in three distinct configurations to address the proof of concept (\pocd). 
The first two configurations involve a random stacking of up to 100k transformations utilizing two separate seed values. 
The third configuration also employs 100k stacked transformations, but it is tailored to the unique needs of the POC attack.
We evaluate the attacks every 10000 stacked mutations.
In total, we generate 3x10 different variants out of the original POC binary.

Given that DCA attacks heavily depend on repetitive patterns found in memory traces, we neutralize the impact of symmetric rewriting rules for the third configuration. 
Specifically, idempotent rewriting rules, like the example one in \autoref{rewriting}, are deactivated. 
This tweaking not only mitigates the effects of the attack but also demonstrates the versatility of \tool as a robust general rewriting tool for \Wasm.

We also make use of IntelPIN to gather \wasm traces during execution with wasmtime, which are then used to feed the DCA attack tool developed by Bos and his colleagues. 
To ensure the purity of our data, we hook into wasmtime to filter out IntelPIN traces, thereby excluding any traces originating from host-specific code within wasmtime.
This step reduces the traces of executing wasmtime in a factor of 100, making the attack easier, thus, making possible the exfiltration of the key from a \Wasm program.

%In addition, this work contributes a novel application: a whitebox cryptography challenge that has been successfully ported to \wasm. This further underscores our commitment to enhancing the capabilities of \Wasm and fortifying its defenses against potential attacks.


\begin{table}[]
    \centering
    \begin{tabular}{l|c | c | c }
        POC & Timing & Mem & Pred. \\
        \hline
         \poct & X &  &  \\ 
         \pocd &  & X &   \\ 
         \pocp & X &  & X \\ 
    \end{tabular}
    \caption{Dataset}
    \label{tab:my_label}
\end{table}

\todo{Discard port contention. In the end it has the previous component on timing.}


\todo{Uncontrolled wasm-mutate makes the inferring of the WBC keys easier sometimes. We already saw this in the  evasion paper. My intuiyion is that if we disable "symetric" transformations, we will obtin better results. Daredevil is looking for "repetitive" patterns, if we use a symetric transformations, we are, indeed, duplicating the binary behavior.}

\todo{The traces might disclose the secret. Yet their collection takes more time due to time overhead. We can call it "traces flooding".}
\todo{The attack time increases considerably. Add a plot of, attack accuracy + performance + attack time vs stacking. Elaborate then in hardening and not only prevention. We have seen the flooding effect with javy, making daredevil to fail in analysisng traces in the order of the Gb. }

\todo{One single key byte takes almost 1 hour with 200k stacked mutations. On the other hand, daredevil fails due to oom.}


\todo{Add only two plots here, the purely random and the controlled.}




\todo{TBD discuss deoptimization}
% \subsection{Deoptimization}

\subsection{Partial input/output validation}

% We need to talk about this because, we do this checking right noe and it is probably a reason for the low count of variants.
When \tool generates a variant, it can be executed to check the input/output equivalence.
If the variant has a \_start function, both binaries, the original and the variant can be initialized. 
If the state of the memory, the globals and the stack is the same after executing the \_start function, they are partially equivalent.
%This mechanismm is already implemented in the fuzzing campaign of wasmtime.

The \_start function is easier to execute given its signature.
It does not receive parameters.
Therefore, it can be executed directly.
Yet, since a \Wasm program might contain more than one function that could be indistinctly called with and arbitrary number of parameters, we are not able to validate the whole program.
Thus, we call the checking of the initialization of a \wasm variant, a partial validation.

\subsection{Some other works to be cited along with the paper. Mostly in the Intro}

\emph{Spectre and side-channel defenses}

- paper 2021: Read this, since it is super related, \url{https://www.isecure-journal.com/article_136367_a3948a522c7c59c65b65fa87571fde7b.pdf} \cite{WasmSpectre}


- A dataset of Wasm programs: \cite{nicholson2023wasmizer}

- Papers 2020

- Papers 2019
- \cite{10.1145/3510003.3510070}

Selwasm: A code protection mechanism for webassembly


Babble

- https://arxiv.org/pdf/2212.04596.pdf

Principled Composition of Function Variants for Dynamic
Software Diversity and Program Protection

- https://dl.acm.org/doi/10.1145/3551349.3559553

How Far We’ve Come – A Characterization Study of Standalone WebAssembly Runtimes

- https://ieeexplore.ieee.org/document/9975423

Code obfuscation against symbolic execution attacks

Code artificiality: A metric for the code stealth based on an n-gram model

Semantics-aware obfuscation scheme prediction for binary

Wobfuscator: Obfuscating javascript malware via opportunistic translation to webassembly

Synthesizing Instruction Selection Rewrite Rules from RTL using SMT
"We also synthesize integer rewrite rules from WebAssembly to RISC-V "

Wafl: Binary-only webassembly fuzzing with fast snapshots



% Move this to conslusions.
%- Cache tming: \cite{}
%The original POC programs are wholly vulnerable to attacks. 
%This risk dramatically escalates when the binary is replicated and potentially executed across a multitude of machines worldwide, such as in a Function-as-a-Service (FaaS) platform like Fastly \cite{MEWE}.
In concrete, distributing the unmodified binary to 100 machines would, essentially, creates 100 homogeneously vulnerable machines.
However, let us illustrate the case with a different approach: each time the binary is replicated onto a different machine, we distribute a unique variant instead of the original binary. 
If we disseminate a unique variant, with X stacked transformations, to each machine, every system would run a distinct \wasm binary. 
Based on our findings, even when some binaries are still vulnerable, we can confidently say that if 100 variants of a vulnerable program, each furnished with X stacked transformations, are distributed, the impact of any potential attack is considerably mitigated.
While it's true that some variants may retain their original vulnerabilities, not all of them do. 
This significantly enhances overall security. 
Further reinforcing this point, let's consider the case of btb\_leakage. 
In this scenario, a suite of 100 variants, each featuring at least 200 stacked transformations, ensures full protection against potential threats, effectively securing the entire infrastructure.
Moreover, considering the results for the ret2spec attack, this property holds for the whole population of generated variants, despite the number of stacked transformations.
Therefore, \tool as a software diversification tool, is a peemptive solution to potential attacks.



This is specially true for V8.
V8 tends to favor rapid compilation over thorough optimizations, resulting in a higher preservation percentage than wasmtime. 
This trend has been noted in earlier studies involving CROW and MEWE, which demonstrated preservation rates of 99\% and 96\% for V8 and wasmtime, respectively, when using CROW diversification. 
Wasmtime, on the other hand, places more emphasis on the optimization of machine code generation.

Comparing preservation rates between CROW and \tool variants reveals that CROW generally achieves higher preservation. 
This can be attributed to CROW's use of a superdiversifier during the synthesis process. 
Despite taking longer, it crafts variants that compilers cannot undo during optimization. 
While a lower preservation rate might not appear advantageous at first glance, our observations highlight that unpreserved variants can put certain compiler components under stress, components that remain dormant during the execution of the original binary. 
In essence, specific optimization implementations may remain idle if the corresponding code isn't included in the \wasm program set for compilation. 
Importantly, despite considering the smallest population size coupled with the lowest preservation percentage, the range of machine codes executed still encompasses thousands of variants, underscoring the effectiveness of diversification strategies.




\begin{metric}{Compiled population size:}\label{metric:popcomp}
Given an original \wasm program P a generated corpus of \wasm programs $V=\{v_1, v_2, ..., v_N\}$ where $v_i$ is a semantically equivalent variant of P and a compiler $C$, the compiled population size is defined as:
$$
    | set(\{ sha256(C(v_1)),..., sha256(C(v_N)) \})|\text{ } \forall v_i \in V 
$$

\end{metric}



% Security issues, CVE, how diversification plays
The \wasm execution model is meticulously designed to be secure, aiming to mitigate a broad spectrum of memory and control flow attacks.
However, as acknowledged in its official documentation \cite{WebAssemblySecurity}, \wasm is not invulnerable to potential exploitations, neither the tools that generate \Wasm programs \cite{usenixWASM2020, stievenart2021security}.
On the flip side, code diversification is a significant proactive that notably enforces security by automatic testing and perturbing security sensible side-channels \cite{arteaga2020crow, MEWE, CABRERAARTEAGA2023103296}.


However, existing diversification techniques are primarily tailored for LLVM compilers, missing potential places in which code diversification can be achieved.
Therefore, as previously mentioned, the emerging technologies that generate \wasm binaries may not fully benefit from these techniques.
As a result, the development of \Wasm to \Wasm diversification tools becomes imperative.
This work presents a new software diversification tool for \Wasm, \tool.
\todo{Why is good to have many variants in just a few minutes. Good for testing and fuzzing.}


% Now, what is new in tool, what does it offer


\todo{Talks about the CVE later.}


%%% Table Dataset

\begin{table}
\renewcommand\arraystretch{1.1}
\begin{adjustbox}{width=\linewidth,totalheight=\textheight, keepaspectratio}
    \begin{tabular}{p{1.2cm} | l | l | r | r | p{2cm} | p{1cm} }
        \hline
        Source & Program & RQ & \#F & \# Ins. & Attack & Comp. \\
        \hline \hline
        CROW & 303 & \ref{rq:static}, \ref{rq:dynamic} & 7-103 & 170-36023 & N/A & C to Wasm \\
        \hline
        Swivel & btb\_breakout & \ref{rq:defensive} & 16 & 743 & Spectre branch target buffer (btb) & manually crafted \\
        \hline
        Swivel & btb\_leakage & \ref{rq:defensive} & 16 & 297 & Spectre branch target buffer(btb) & manually crafted \\
        \hline
        Safeside & ret2spec & \ref{rq:defensive} & 2977 & 378894 & Spectre Return Stack Buffer (rsb) & C to Wasm \\
        \hline
        Safeside & pht & \ref{rq:defensive} & 2978 & 379058 & Spectre Pattern History Table (pht) & C to Wasm \\

%\end{adjustbox}
    \end{tabular}
\end{adjustbox}
    
    \caption{\wasm dataset used to evaluate \tool. Each row in the table corresponds to programs, with the columns providing: where the program is sourced from, the number of programs, research question addressed, function count, the total number of instructions found in the original \wasm program, the type of attack that the original program was subjected to and the description of the process employed to convert the source code into the corresponding \wasm programs.}
    \label{tab:corpus}
\end{table}



% Words on the workload of each program
\todo{different input = different trace is a truism, which might be what would be understood by the reader. I see where you want to go but it's too subtle, and it highlights the weakness of the dataset (no input), so I would remove }
Notice that, each program and its variants are executed with their initial input.
In practice, the traces we collect are generated from instantiating the \wasm binary and then executing its main function. 
For most programs in the CROW suite, their main function carries out a specific algorithm (for instance, \emph{base64 encoding}) using a simple, single-input model. 
This approach helps to collect execution traces without crafting programs input.
Yet, it potentially reduce the uniqueness of the traces collected. 
For instance, sections of the program subject to conditional execution might not be activated, not necessarily due to their 'dead-code' status, but rather due to the limited variety of inputs. 
However, even with this relatively modest input-testing approach, \tool still succeeds in providing more than one unique execution trace for each program and its variants population.
it is worth noting that the ratio of unique traces would likely increase given a broader input spectrum.


\todo{we have not provided any number, so we should substantiate this with a claim }
The substantial differences in code preservation between JIT compilers carry notable implications. 
\todo{not a sentence}
For example, between wasmtime and V8.
On one side, high preservation rates signal a possible deficiency in optimizations, as previously underlined in \cite{wasmslim}. 
Conversely, high preservation may inadvertently allow vulnerable code to persist.
For instance, although \tool can generate more resilient variants, the compiler's role in mitigating initial vulnerabilities is crucial. 
High preservation could, therefore, potentially hinder the removal of potential security weaknesses during the compilation process.
\todo{subsection not convincing I would remove}


\subsection{Program Normalization}
\todo{this paragraph is not at all about wasm-mutate, remove}
\tool was previously employed successfully for the evasion of malware detection, as outlined in \cite{CABRERAARTEAGA2023103296}. 
The proposed mitigation in the prior study involved code normalization as a means of reducing the spectrum of malware variants. 
Our current work provides insights into the potential effectiveness of this approach. 
Specifically, a practically costless process of pre-compiling Wasm binaries could be employed as a preparatory measure for malware classifiers. 
In other words, a \wasm binary can first be compiled with wasmtime, effectively eliminating approx. 25\% of malware variants according to our preservation statistics for wasmtime. 
This approach could substantially enhance the efficiency and precision of malware detection systems.


This paper unveils \tool, an innovative, compiler-independent method for \Wasm software diversification. \tool, built atop an e-graph, encompasses 135 artfully designed rewriting rules catering to diversification from granular to broad levels. 
Our approach enables the generation of ten of thousands of equivalent program variants within minutes. 
\tool excels in swiftly creating a vast array of unique, highly preserved \wasm programs, demonstrating significant variations in execution traces, including memory. 
Notably, \tool enhances the resilience of \wasm programs initially vulnerable to Spectre attacks, thereby offering a promising defensive strategy against such threats. 
Hence, \tool, with its contribution to the wider \wasm ecosystem, paves the way for safer and more robust \wasm applications.