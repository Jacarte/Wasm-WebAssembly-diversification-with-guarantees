
\lstdefinestyle{watcode}{
  numbers=none,
  stepnumber=1,
  numbersep=10pt,
  tabsize=4,
  showspaces=false,
  breaklines=true, 
  showstringspaces=false,
  moredelim=**[is][{\btHL[fill=black!10]}]{`}{`},
  moredelim=**[is][{\btHL[fill=celadon!40]}]{!}{!}
}
{
%\captionsetup{width=0.9\linewidth}
\lstset{
  language=WAT,
  style=watcode,
  breaklines=true, 
  basicstyle=\footnotesize\ttfamily,
  %numberstyle=\footnotesize,
  numbersep=2.5pt,
  %firstnumber=1,
  escapeinside={``},
  %numbers=left,
  %postbreak=\mbox{\textcolor{red}{$\hookrightarrow$}\space},
  }
    \begin{lstlisting}[label=example:wasmprogram,caption={Simplified WebAssembly code for the program of \autoref{example:cprogram}.}, captionpos=b]{Name}
(module
  (@custom "producer" "llvm.." )
  (import "env" "println" (func $println (param i32)))
  (memory 1)
  (export "memory" (memory 0))
  (func $main
    (local $sum i32)
    (local $i i32)
    (local $arr_offset i32)
    ; Initialize sum to 0 ;
    i32.const 0
    local.set $sum
    ; Initialize arr_offset to point to start of the array in memory ;
    i32.const 0
    local.set $arr_offset
    ; Initialize the array in memory;
    i32.const 0
    i32.const 1
    i32.store
    ...
    i32.store
    ...
    loop
      local.get $i
      i32.const 5
      i32.lt_s
      if
        ; Load array[i] and add to sum ;
        local.get $arr_offset
        local.get $i
        ...
        ; Increment i ;
        local.get $i
        i32.const 1
        i32.add
        local.set $i
        br 0
      else
        ; End loop ;
        i32.const 0
      end
    end
    
    ; Call external function to print sum ;
    local.get $sum
    call $println
  )
  ; Start the main function ;
  (start $main)
  )
\end{lstlisting}
}
