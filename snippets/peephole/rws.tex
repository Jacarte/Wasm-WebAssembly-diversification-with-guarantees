We include well-known identity reduction rewriting rules. 
We exemplify them in the following four rewriting rules. 
Observe that the primitive type of the rewriting rules can be changed to the other remaining types. 


\begin{minipage}{0.95\linewidth}
\begin{minipage}{0.49\linewidth}
    
    \lstset{
    language=ttt,
    style=watcode,
    basicstyle=\footnotesize\ttfamily,
    columns=fullflexible,
    breaklines=true}
    \begin{lstlisting}[]
LHS i32.or ?x i32.const.-1
            \end{lstlisting}\vspace{-0.5cm}
    \noindent\hrulefill
        \lstset{
            language=ttt,
            style=watcode,
            basicstyle=\footnotesize\ttfamily,
            columns=fullflexible,
            breaklines=true}
            \vspace{-0.2cm}
            \begin{lstlisting}[numbers=none]{Name}
RHS i32.const.-1
    \end{lstlisting}
\end{minipage}
\begin{minipage}{0.49\linewidth}
    \lstset{
    language=ttt,
    style=watcode,
    basicstyle=\footnotesize\ttfamily,
    columns=fullflexible,
    breaklines=true}
    \begin{lstlisting}[]
LHS f32.mul ?x f32.const.1
            \end{lstlisting}\vspace{-0.5cm}
    \noindent\hrulefill
        \lstset{
            language=ttt,
            style=watcode,
            basicstyle=\footnotesize\ttfamily,
            columns=fullflexible,
            breaklines=true}
            \vspace{-0.2cm}
            \begin{lstlisting}[numbers=none]{Name}
RHS ?x
    \end{lstlisting}
\end{minipage}    
\end{minipage}


\begin{minipage}{0.95\linewidth}
\begin{minipage}{0.49\linewidth}
    
    \lstset{
    language=ttt,
    style=watcode,
    basicstyle=\footnotesize\ttfamily,
    columns=fullflexible,
    breaklines=true}
    \begin{lstlisting}[]
LHS i32.eq ?x i32.const.0
            \end{lstlisting}\vspace{-0.5cm}
    \noindent\hrulefill
        \lstset{
            language=ttt,
            style=watcode,
            basicstyle=\footnotesize\ttfamily,
            columns=fullflexible,
            breaklines=true}
            \vspace{-0.2cm}
            \begin{lstlisting}[numbers=none]{Name}
RHS i32.eqz ?x
    \end{lstlisting}
\end{minipage}
\begin{minipage}{0.49\linewidth}
    \lstset{
    language=ttt,
    style=watcode,
    basicstyle=\footnotesize\ttfamily,
    columns=fullflexible,
    breaklines=true}
    \begin{lstlisting}[]
LHS i64.shl ?x i64.const.0
            \end{lstlisting}\vspace{-0.5cm}
    \noindent\hrulefill
        \lstset{
            language=ttt,
            style=watcode,
            basicstyle=\footnotesize\ttfamily,
            columns=fullflexible,
            breaklines=true}
            \vspace{-0.2cm}
            \begin{lstlisting}[numbers=none]{Name}
RHS ?x
    \end{lstlisting}
\end{minipage}    
\end{minipage}


We have also incorporated well-known commutative rewriting rules. 
THe following three rules provide examples. 
As discussed earlier, observe that the subexpressions' type can be changed to any of the original four primitive \Wasm data types.

\begin{minipage}{0.95\linewidth}
\begin{minipage}{0.49\linewidth}
    
    \lstset{
    language=ttt,
    style=watcode,
    basicstyle=\footnotesize\ttfamily,
    columns=fullflexible,
    breaklines=true}
    \begin{lstlisting}[]
LHS select ?y ?y ?x
            \end{lstlisting}\vspace{-0.5cm}
    \noindent\hrulefill
        \lstset{
            language=ttt,
            style=watcode,
            basicstyle=\footnotesize\ttfamily,
            columns=fullflexible,
            breaklines=true}
            \vspace{-0.2cm}
            \begin{lstlisting}[numbers=none]{Name}
RHS ?y
    \end{lstlisting}
\end{minipage}
\begin{minipage}{0.49\linewidth}
    \lstset{
    language=ttt,
    style=watcode,
    basicstyle=\footnotesize\ttfamily,
    columns=fullflexible,
    breaklines=true}
    \begin{lstlisting}[]
LHS i32.add ?x ?y
            \end{lstlisting}\vspace{-0.5cm}
    \noindent\hrulefill
        \lstset{
            language=ttt,
            style=watcode,
            basicstyle=\footnotesize\ttfamily,
            columns=fullflexible,
            breaklines=true}
            \vspace{-0.2cm}
            \begin{lstlisting}[numbers=none]{Name}
RHS i32.add ?y ?x
    \end{lstlisting}
\end{minipage}    
\end{minipage}


\begin{minipage}{0.95\linewidth}
\begin{minipage}{0.49\linewidth}
    
    \lstset{
    language=ttt,
    style=watcode,
    basicstyle=\footnotesize\ttfamily,
    columns=fullflexible,
    breaklines=true}
    \begin{lstlisting}[]
LHS i32.mul ?x ?y
            \end{lstlisting}\vspace{-0.5cm}
    \noindent\hrulefill
        \lstset{
            language=ttt,
            style=watcode,
            basicstyle=\footnotesize\ttfamily,
            columns=fullflexible,
            breaklines=true}
            \vspace{-0.2cm}
            \begin{lstlisting}[numbers=none]{Name}
RHS i32.mul ?y ?x
    \end{lstlisting}
\end{minipage}
\begin{minipage}{0.49\linewidth}
\end{minipage}    
\end{minipage}


In the following 2 rewriting rules we exemplify two associative rewriting rules.


\begin{minipage}{0.95\linewidth}
\begin{minipage}{0.49\linewidth}
    
    \lstset{
    language=ttt,
    style=watcode,
    basicstyle=\footnotesize\ttfamily,
    columns=fullflexible,
    breaklines=true}
    \begin{lstlisting}[]
LHS i32.mul ?x (i32.mul ?y ?z)
            \end{lstlisting}\vspace{-0.5cm}
    \noindent\hrulefill
        \lstset{
            language=ttt,
            style=watcode,
            basicstyle=\footnotesize\ttfamily,
            columns=fullflexible,
            breaklines=true}
            \vspace{-0.2cm}
            \begin{lstlisting}[numbers=none]{Name}
RHS i32.mul (i32.mul ?x ?y) ?z
    \end{lstlisting}
\end{minipage}
\begin{minipage}{0.49\linewidth}
    \lstset{
    language=ttt,
    style=watcode,
    basicstyle=\footnotesize\ttfamily,
    columns=fullflexible,
    breaklines=true}
    \begin{lstlisting}[]
LHS i32.add ?x (i32.add ?y ?z)
            \end{lstlisting}\vspace{-0.5cm}
    \noindent\hrulefill
        \lstset{
            language=ttt,
            style=watcode,
            basicstyle=\footnotesize\ttfamily,
            columns=fullflexible,
            breaklines=true}
            \vspace{-0.2cm}
            \begin{lstlisting}[numbers=none]{Name}
RHS i32.add (i32.add ?x ?y) ?z
    \end{lstlisting}
\end{minipage}    
\end{minipage}

We have also incorporated several strength reduction rewriting rules.
We will exemplify them in the following text.
Note that the extent of these rewriting rules can be adjusted as desired.
For instance, although we only include reductions up to multiplication by 8, they can be extended to any power of 2.

\begin{minipage}{0.95\linewidth}
\begin{minipage}{0.49\linewidth}
    
    \lstset{
    language=ttt,
    style=watcode,
    basicstyle=\footnotesize\ttfamily,
    columns=fullflexible,
    breaklines=true}
    \begin{lstlisting}[]
LHS i32.shl ?x i32.const.1
            \end{lstlisting}\vspace{-0.5cm}
    \noindent\hrulefill
        \lstset{
            language=ttt,
            style=watcode,
            basicstyle=\footnotesize\ttfamily,
            columns=fullflexible,
            breaklines=true}
            \vspace{-0.2cm}
            \begin{lstlisting}[numbers=none]{Name}
RHS i32.mul ?x i32.const.2
    \end{lstlisting}
\end{minipage}
\begin{minipage}{0.49\linewidth}
    \lstset{
    language=ttt,
    style=watcode,
    basicstyle=\footnotesize\ttfamily,
    columns=fullflexible,
    breaklines=true}
    \begin{lstlisting}[]
LHS i32.shl ?x i32.const.3
            \end{lstlisting}\vspace{-0.5cm}
    \noindent\hrulefill
        \lstset{
            language=ttt,
            style=watcode,
            basicstyle=\footnotesize\ttfamily,
            columns=fullflexible,
            breaklines=true}
            \vspace{-0.2cm}
            \begin{lstlisting}[numbers=none]{Name}
RHS i32.mul ?x i32.const.8
    \end{lstlisting}
\end{minipage}    
\end{minipage}


\begin{minipage}{0.95\linewidth}
\begin{minipage}{0.49\linewidth}
    
    \lstset{
    language=ttt,
    style=watcode,
    basicstyle=\footnotesize\ttfamily,
    columns=fullflexible,
    breaklines=true}
    \begin{lstlisting}[]
LHS i32.add ?x ?x
            \end{lstlisting}\vspace{-0.5cm}
    \noindent\hrulefill
        \lstset{
            language=ttt,
            style=watcode,
            basicstyle=\footnotesize\ttfamily,
            columns=fullflexible,
            breaklines=true}
            \vspace{-0.2cm}
            \begin{lstlisting}[numbers=none]{Name}
RHS i32.mul ?x i32.const.2
    \end{lstlisting}
\end{minipage}
\begin{minipage}{0.49\linewidth}
\end{minipage}    
\end{minipage}





\begin{minipage}{0.95\linewidth}
\begin{minipage}{0.49\linewidth}
    
    \lstset{
    language=ttt,
    style=watcode,
    basicstyle=\footnotesize\ttfamily,
    columns=fullflexible,
    breaklines=true}
    \begin{lstlisting}[]
LHS drop ?x
            \end{lstlisting}\vspace{-0.5cm}
    \noindent\hrulefill
        \lstset{
            language=ttt,
            style=watcode,
            basicstyle=\footnotesize\ttfamily,
            columns=fullflexible,
            breaklines=true}
            \vspace{-0.2cm}
            \begin{lstlisting}[numbers=none]{Name}
RHS drop i32.rand
    \end{lstlisting}
\end{minipage}
\begin{minipage}{0.49\linewidth}
\end{minipage}    
\end{minipage}


As previously mentioned, our work is built on well-established diversification strategies. 
In the subsequent rewriting rules, we exemplify the porting of two well-known strategies. 
The leftmost part illustrates the injection of \texttt{nop} instructions. 
The rightmost part demonstrates the "un-folding" of constants. 
In the latter case, statically defined constants are substituted by the sum of two numbers. 
This sum computes the original constant at runtime.


\begin{minipage}{0.95\linewidth}
\begin{minipage}{0.49\linewidth}
    
    \lstset{
    language=ttt,
    style=watcode,
    basicstyle=\footnotesize\ttfamily,
    columns=fullflexible,
    breaklines=true}
    \begin{lstlisting}[]
LHS ?x
            \end{lstlisting}\vspace{-0.5cm}
    \noindent\hrulefill
        \lstset{
            language=ttt,
            style=watcode,
            basicstyle=\footnotesize\ttfamily,
            columns=fullflexible,
            breaklines=true}
            \vspace{-0.2cm}
            \begin{lstlisting}[numbers=none]{Name}
RHS (nop ?x)
    \end{lstlisting}
\end{minipage}
\begin{minipage}{0.49\linewidth}
    \lstset{
    language=ttt,
    style=watcode,
    basicstyle=\footnotesize\ttfamily,
    columns=fullflexible,
    breaklines=true}
    \begin{lstlisting}[]
LHS ?x
            \end{lstlisting}\vspace{-0.5cm}
    \noindent\hrulefill
        \lstset{
            language=ttt,
            style=watcode,
            basicstyle=\footnotesize\ttfamily,
            columns=fullflexible,
            breaklines=true}
            \vspace{-0.2cm}
            \begin{lstlisting}[numbers=none]{Name}
RHS i32.add (i32.const z i32.const y) 
    \end{lstlisting}\vspace{-0.5cm}
    \noindent\hrulefill
\lstset{
        language=ttt,
        style=watcode,
        basicstyle=\footnotesize\ttfamily,
        columns=fullflexible,
        breaklines=true}
        \begin{lstlisting}[numbers=none]{Name}
Cond z = i32 random & y = x - z 
        \end{lstlisting}
\end{minipage}    
\end{minipage}


The implemented rewriting rules can be employed commutatively. 
For example, the left-hand side (LHS) can function interchangeably as the right-hand side (RHS), and the reverse is also valid. 
Consequently, observe the doubling of practical rewriting occurrences. 
Modern compilers can generally reverse peephole rewriting rules.  
This also facilitates manual verification. 
Despite the ability of the compiler to reverse these rules, it becomes significantly more challenging when utilizing an e-graph. 
The primary cause of this difficulty is the interleaving of the rewriting rules within the e-graph.

