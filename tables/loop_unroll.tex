Loop unroll
&  
        \vspace{-4mm}\begin{lstlisting}[numbers=none,escapechar=?]{Name}
        (loop ?\tikzmarkPROBE{2}{bb2}{15}{7}?   ?\tikzmarkPROBE{2}{bb2}{1}{2}? 
bb A bb 
            br_if 0 ?\tikzmarkPROBE{1}{bb1}{-0 .2}{1.5}?
bc B bc 
        end) ?\tikzmarkPROBE{3}{bb1}{1}{1.5}?
?\tikzmarkPROBE{4}{bb1}{15}{1.5}?  \end{lstlisting}   
&  
        \vspace{-4mm}
\begin{lstlisting}[numbers=none,escapechar=?]{Name}
        (block (block
bb A' bb ?\tikzmarkPROBE{11}{bb1}{-13}{2}?
                br_if 0 ?\tikzmarkPROBE{9}{bb1}{-2}{2}?
bc B' bc ?\tikzmarkPROBE{7}{bb1}{-13}{2}?
                br 1  ?\tikzmarkPROBE{7}{bb1}{-1.3}{4}?
            end) ?\tikzmarkPROBE{10}{bb1}{2}{2}?
            (loop ?\tikzmarkPROBE{5}{bb1}{2}{2}?
bb A' bb 
             br_if 0 ?\tikzmarkPROBE{6}{bb1}{-3.5}{1}?
bc B' bc 
            end) ?\tikzmarkPROBE{11}{bb1}{-1}{2}?
        end) ?\tikzmarkPROBE{8}{bb1}{5}{2}?
    )\end{lstlisting}  
&   C \\
        \vspace{-4mm} \\
%& \multicolumn{3}{l}{ 
%\begin{minipage}{0.85\textwidth}

 %   \begin{tcolorbox}[boxrule=0.3pt,arc=.1em,boxsep=-1.5mm]
 %       \textbf{Guarantees:} Any jump instructions within \texttt{A'} and \texttt{B'} that originally jumped outside the loop will have their jump index increased by one. Besides, there is an unconditional branch at the end of the unrolled loop iteration's body. This ensures that if the loop body doesn't continue, the tool breaks out of the scope instead of continuing to the not-unrolled loop.
 %   \end{tcolorbox}
%\end{minipage}
%} \\
%&
%$[St, S, L, Mi, Fi, Ti, M, G, E, I]$ &
%$[St, S, L, Mi, Fi, Ti, M, G, E, I]$ \\
\hline



\begin{tikzpicture}[remember picture,overlay]
%\path (2.east) edge[<-, bend left, dashed] (1.east);
%\path (2.east) edge[<-, bend left, dashed] (3.east);
\path (1.west) edge[->, bend right, dashed] (2.west);
\path (3.east) edge[->, bend left, dashed] (4.east);
\end{tikzpicture}


\begin{tikzpicture}[remember picture,overlay]
\path (6.west) edge[->, bend right, dashed] (5.west);
\path (7.west) edge[->, bend left, dashed] (8.west);
\path (9.west) edge[->, bend left, dashed] (10.west);
\path (11.west) edge[->, bend left, dashed] (8.west);
%\path (10.east) edge[<-, bend left, dashed] (6.east);
%\path (10.east) edge[<-, bend left, dashed] (7.east);
\end{tikzpicture}